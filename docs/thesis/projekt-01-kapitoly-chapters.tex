
%===========================================================================================
%	ÚVODNÍ KAPITOLA
%
% - 1-2 strany textu
% - co nejlépe čitelný pro kohokoliv
% - seznámení s prací
% - krátké seznámení s odvětvím (v rámci IT - grafika, ...)
%===========================================================================================
\chapter{Úvod}
\label{ch:uvod}

\todo{Úvod do tématu - obecně známá slova, nejlépe bez odborných výrazů}

\todo{Cíl práce - o co se pokoušíme, co má být výsledkem práce}

\todo{Struktura práce - co se kde probírá (odkazy na kapitoly, strany)}


%===========================================================================================
%	KAPITOLA TEORIE
%
% - 40-50% obsahu práce
% - vhodné psát vlastními slovy, obsahově může kopírovat literaturu
% - trohu znalý člověk z oblasti by měl být schopen vše pochopit
%===========================================================================================
\chapter{Teorie}
\label{ch:teorie}

\section{Vizualizace}

\section{Perlinův šum}


%===========================================================================================
%	KAPITOLA NÁVRHU ŘEŠENÍ NÁSTROJE / SHRNUTÍ DOSAVADNÍHO STAVU OBLASTI
%
% - smyslem je na základě zhodnocení současného stavu určit cíl práce
% - obecně platný (není konkrétní k C++ či OpenGL, ...)
% - odkazy na teoretickou část
%===========================================================================================
\chapter{Návrh řešení}
\label{ch:navrh}

\todo{Co už existuje, co už je vyřešeno.}

\todo{Co by bylo vhodné vyřešit.}

\todo{Co to má dělat?}
\todo{Jaké to má mít parametry?}


%===========================================================================================
%	KAPITOLA IMPLEMENTACE NÁSTROJE
%
% - co tvoří podstatu vlastní práce
% - jak a jakými prostředky bylo dílo vytvořeno
% - jaké jsou výsledky práce
% - veškeré technické detaily, které nejsou podstatné k pochopení podstaty práce patří do příloh
%===========================================================================================
\chapter{Implementace}
\label{ch:implementace}

\todo{Blokové schéma, objektový návrh, seznam tříd, struktura SW...}

\todo{Volba OS...}
\todo{Volba programovacího jazyka...}
\todo{Volba knihoven...}


\todo{Omezení, problémy SW...}


%===========================================================================================
%	ZÁVĚREČNÁ KAPITOLA
%
%===========================================================================================
\chapter{Závěr}
\label{ch:zaver}

\todo{Krátké shrnutí závěru práce...}

\todo{Jak byl záměr práce splněn?}
\todo{Zhodnocení dosažených výsledků.}

\todo{Co nebylo implementováno, možnost pokračování, výhled do budoucna.}

\todo{Osobní dojmy - co mi práce dala?}





