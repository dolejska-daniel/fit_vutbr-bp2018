
%===========================================================================================
%	ÚVODNÍ KAPITOLA
%
% - 1-2 strany textu
% - co nejlépe čitelný pro kohokoliv
% - seznámení s prací
% - krátké seznámení s odvětvím (v rámci IT - grafika, ...)
%===========================================================================================
\chapter{Úvod}\label{ch:uvod}
\todo{Úvod do tématu - obecně známá slova, nejlépe bez odborných výrazů}

\todo{Cíl práce - o co se pokoušíme, co má být výsledkem práce}

\todo{Struktura práce - co se kde probírá (odkazy na kapitoly, strany)}


%===========================================================================================
%	KAPITOLA TEORIE
%
% - 40-50% obsahu práce
% - vhodné psát vlastními slovy, obsahově může kopírovat literaturu
% - trohu znalý člověk z oblasti by měl být schopen vše pochopit
%===========================================================================================
\chapter{\todo{Teorie}}\label{ch:teorie}
\todo{Co je obsahem této sekce...}


\section{OpenGL}\label{ch:teorie:opengl}
\todo{Co je to OpenGL, k čemu se používá?}

\subsection{Vertex}\label{ch:teorie:opengl:vertex}
\todo{Co je to vertex?}

\begin{figure}[H]
	\centering
	\begin{subfigure}{0.32\textwidth}
		\includegraphics[width=\textwidth]{obrazky-figures/placeholder.pdf}
		\caption{Pozice}
		\label{fig:opengl:vertex:p}
	\end{subfigure}
	\hfill
	\begin{subfigure}{0.32\textwidth}
		\includegraphics[width=\textwidth]{obrazky-figures/placeholder.pdf}
		\caption{Pozice, Normála}
		\label{fig:opengl:vertex:pn}
	\end{subfigure}
	\hfill
	\begin{subfigure}{0.32\textwidth}
		\includegraphics[width=\textwidth]{obrazky-figures/placeholder.pdf}
		\caption{Pozice, Normála, Barva}
		\label{fig:opengl:vertex:pnb}
	\end{subfigure}
	\caption{Ukázka vertexů s různými atributy}
	\label{fig:opengl:vertex}
\end{figure}

\todo{K čemu se používá?}
\todo{Jak a kde se používá?}

\subsubsection*{Vertex atribut}\label{ch:teorie:opengl:vertex:attr}
\todo{Co je to vertex?}
\todo{Jak a kde se používá?}

\subsection{Index}\label{ch:teorie:opengl:index}
\todo{Co je to index?}
\todo{Jak a kde se používá?}

\subsection{Shadery}\label{ch:teorie:opengl:shadery}
\todo{Co je to shader?}
\todo{Jak a kde se používá?}

\subsubsection*{Vertex shader}\label{ch:teorie:opengl:vertex-shader}
\todo{Co konkrétně dělá vertex shader?}

\begin{figure}[H]
	\centering
	\includegraphics[width=.5\linewidth]{obrazky-figures/placeholder.pdf}
	\caption{Demonstrace funkce vertex shaderu}
	\label{fig:opengl:vertex-shader}
\end{figure}

\subsubsection*{Fragment shader}\label{ch:teorie:opengl:fragment-shader}
\todo{Co konkrétně dělá fragment shader?}

\begin{figure}[H]
	\centering
	\includegraphics[width=.5\linewidth]{obrazky-figures/placeholder.pdf}
	\caption{Demonstrace funkce fragment shaderu}
	\label{fig:opengl:vertex-shader}
\end{figure}


\section{Lineární algebra}\label{ch:teorie:lineární-algebra}
\todo{XXXXXXXXXXXXXXXXXXXXXX}

\subsection{Vektory}\label{ch:teorie:lineární-algebra:vektory}
\todo{XXXXXXXXXXXXXXXXXXXXXX}

\begin{figure}[H]
	\centering
	\includegraphics[width=.5\linewidth]{obrazky-figures/placeholder.pdf}
	\caption{Vektor}
	\label{fig:lineární-algebra:vektor}
\end{figure}

\subsubsection*{Sčítání}
\todo{Jak funguje sčítání vektorů?}
\todo{Kdy je užitečné a kdy se používá? Příklad.}

\begin{figure}[H]
	\centering
	\includegraphics[width=.5\linewidth]{obrazky-figures/placeholder.pdf}
	\caption{Sčítání vektorů}
	\label{fig:lineární-algebra:sčítání}
\end{figure}

\subsubsection*{Odčítání}
\todo{Jak funguje odečítání vektorů?}
\todo{Kdy je užitečné a kdy se používá? Příklad.}

\begin{figure}[H]
	\centering
	\includegraphics[width=.5\linewidth]{obrazky-figures/placeholder.pdf}
	\caption{Odčítání vektorů}
	\label{fig:lineární-algebra:odčítání}
\end{figure}

\subsubsection*{\todo{Dot product}}
\todo{Co je to DOT product a jak to funguje?}
\todo{Kdy je užitečné a kdy se používá? Příklad.}

\begin{figure}[H]
	\centering
	\includegraphics[width=.5\linewidth]{obrazky-figures/placeholder.pdf}
	\caption{DOT product}
	\label{fig:lineární-algebra:dot}
\end{figure}

\subsubsection*{\todo{Cross product}}
\todo{Co je to CROSS product a jak to funguje?}
\todo{Kdy je užitečné a kdy se používá? Příklad.}

\begin{figure}[H]
	\centering
	\includegraphics[width=.5\linewidth]{obrazky-figures/placeholder.pdf}
	\caption{DOT product}
	\label{fig:lineární-algebra:cross}
\end{figure}

\subsection{Průsečíky}\label{ch:teorie:lineární-algebra:průsečíky}
\todo{XXXXXXXXXXXXXXXXXXXXXX}


\section{Perlinův šum}\label{ch:teorie:perlin}
\todo{Co je to Perlinův šum a čím je speciální?}
\todo{Jaké existují další typy šumů?}

\begin{figure}[H]
	\centering
	\begin{subfigure}{0.32\textwidth}
		\includegraphics[width=\textwidth]{obrazky-figures/placeholder.pdf}
		\caption{Náhodný šum}
		\label{fig:perlin:porovnání:náhodný}
	\end{subfigure}
	\hfill
	\begin{subfigure}{0.32\textwidth}
		\includegraphics[width=\textwidth]{obrazky-figures/placeholder.pdf}
		\caption{Perlinův šum}
		\label{fig:perlin:porovnání:perlin}
	\end{subfigure}
	\hfill
	\begin{subfigure}{0.32\textwidth}
		\includegraphics[width=\textwidth]{obrazky-figures/placeholder.pdf}
		\caption{Simplexní šum}
		\label{fig:perlin:porovnání:simplex}
	\end{subfigure}
	\caption{Porovnání náhodného, Perlinova a simplexního šumu}
	\label{fig:perlin:porovnání}
\end{figure}


\section{Textury}\label{ch:teorie:textury}
\todo{XXXXXXXXXXXXXXXXXXXXXX}


%===========================================================================================
%	KAPITOLA NÁVRHU ŘEŠENÍ NÁSTROJE / SHRNUTÍ DOSAVADNÍHO STAVU OBLASTI
%
% - smyslem je na základě zhodnocení současného stavu určit cíl práce
% - obecně platný (není konkrétní k C++ či OpenGL, ...)
% - odkazy na teoretickou část
%===========================================================================================
\chapter{Návrh řešení}\label{ch:navrh}

\todo{Co už existuje, co už je vyřešeno.}

\todo{Co by bylo vhodné vyřešit.}

\todo{Co to má dělat?}
\todo{Jaké to má mít parametry?}


%===========================================================================================
%	KAPITOLA IMPLEMENTACE NÁSTROJE
%
% - co tvoří podstatu vlastní práce
% - jak a jakými prostředky bylo dílo vytvořeno
% - jaké jsou výsledky práce
% - veškeré technické detaily, které nejsou podstatné k pochopení podstaty práce patří do příloh
%===========================================================================================
\chapter{Implementace}\label{ch:impl}

\todo{Blokové schéma, objektový návrh, seznam tříd, struktura SW...}

\todo{Volba OS...}
\todo{Volba programovacího jazyka...}
\todo{Volba knihoven...}

\todo{Omezení, problémy SW...}


\section{Použité knihovny}\label{ch:impl:knihovny}
\todo{geGL}
\todo{ovládání OpenGL}

\todo{ArgumentViewer}
\todo{zpracování příkazů}

\todo{BasicCamera}
\todo{snadná správa kamery}
\todo{transformační matice, ...}

\todo{SDL2CPP}
\todo{SDL wrapper pro C++}

\todo{Vars}
\todo{správa proměnných napříč programem}

\todo{glm}
\todo{matematická knihovna}
\todo{"implementuje" funkce z shaderů}

\todo{SDL2}
\todo{komunikace s Windows API}
\todo{sprava okna}
\todo{správa input (keyboard, mouse)}


\section{Zobrazení}\label{ch:impl:zobrazení}
\todo{Jak probíhá vykreslování?}
\todo{Klíčové vlastnosti?}


\section{Terén}\label{ch:impl:terén}
\todo{Jaký je cíl, co má být výsledkem generování?}

\todo{Potřebné třídy pro generování - map, chunk, heightMap}

\todo{Samotný proces generování}
\todo{vytváření vrcholů}
\todo{vytváření normál}
\todo{vytváření indexů}

\todo{Použité struktury}

\todo{Existující problémy, podněty pro zlepšení?}


\section{Silnice}\label{ch:impl:silnice}
\todo{Jaký je cíl, co má být výsledkem generování?}

\todo{Potřebné třídy pro generování - streetMap, street, heightMap, map}

\todo{Samotný proces generování}
\todo{vytváření podsilnic}
\todo{rozšiřování silnic ("prodlužování")}
\todo{výpočet kolizí (+ QuadTree)}
\todo{ukládání kolizí a práce s nimi}

\todo{Existující problémy, podněty pro zlepšení?}


\section{Parcely}\label{ch:impl:parcely}
\todo{Jaký je cíl, co má být výsledkem generování?}
\todo{parcely pro budovy}
\todo{parcely pro silnice}

\todo{Potřebné třídy pro generování - streetMap, street}

\todo{Samotný proces generování}
\todo{průchod přes silnice}
\todo{vyhledání a použití "křižovatek"}
\todo{neduplicita parcel}


\section{Budovy}\label{ch:impl:budovy}
\todo{Jaký je cíl, co má být výsledkem generování?}
\todo{typy budov}

\todo{Potřebné třídy pro generování - parcel, building, buildingPart, heightMap}

\todo{Samotný proces generování}
\todo{vytváření kvádrů/jiných tvarů}
\todo{jejich posun/orientace}
\todo{generování normál}


\section{Textury}\label{ch:impl:textury}

\subsection{Térén}\label{ch:impl:textury:terén}

\subsection{Budovy}\label{ch:impl:textury:budovy}

\subsection{Prostředí}


%===========================================================================================
%	ZÁVĚREČNÁ KAPITOLA
%
%===========================================================================================
\chapter{Závěr}\label{ch:zaver}

\todo{Krátké shrnutí závěru práce...}

\todo{Jak byl záměr práce splněn?}
\todo{Zhodnocení dosažených výsledků.}

\todo{Co nebylo implementováno, možnost pokračování, výhled do budoucna.}

\todo{Osobní dojmy - co mi práce dala?}





