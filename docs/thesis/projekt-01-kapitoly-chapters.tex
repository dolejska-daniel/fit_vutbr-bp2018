
%===========================================================================================
%	ÚVODNÍ KAPITOLA
%
% - 1-2 strany textu
% - co nejlépe čitelný pro kohokoliv
% - seznámení s prací
% - krátké seznámení s odvětvím (v rámci IT - grafika, ...)
%===========================================================================================
\chapter{Úvod}\label{ch:uvod}
%\todo{Úvod do tématu - obecně známá slova, nejlépe bez odborných výrazů (5ř)}
S~využitím procedurálního generování v~různých oblastech počítačové grafiky se můžeme setkat čím dál tím častěji.
Cílem tohoto přístupu je zprostředkovat relativně snadné vytvoření velkého množství obsahu s~různými předem stanovenými vlastnostmi.
Je jej využíváno především v~herním a~filmovém průmyslu.

Ve filmovém průmyslu může být těchto principů využito například pro vytvoření komplexní a~rozsáhlé scény.
Typickým příkladem tak může být scéna na pozadí\,--\,ať už se jedná o~městskou čtvrť, deštný prales či dechberoucí planetární scenérie ve vesmíru.
V~herním průmyslu může být procedurálního generování využito k~zajištění vysoké různorodosti prostředí a~to i~za použití ručně vytvořených modelů.
V neposlední řadě může sloužit k~vytváření celých herních světů, kde i~ty nejmenší detaily mohou být snadno nastavitelné.

\todo{Proč je práce důležitá a jaký má význam?}
\todo{Jaký je/byl stav v této oblasti?}
Správným použitím procedurálního generování obsahu můžeme potenciálně ušetřit velké množství času, který bychom jinak museli investovat do jeho ručního vytváření.
V dnešní době existují komerční i~open source programy, které poskytují různé nástroje pro procedurální generování terénu i jiných modelů.
\todo{Nástroj má specifické využití, opensource verze neexistuje.}

%\todo{Cíl práce - o co se pokoušíme, co má být výsledkem práce}
Cílem této práce je navrhnout a~implementovat nástroj, pomocí kterého bude možné sestavit a~exportovat model města.
Jeho hlavním rysem by měla být jeho konfigurovatelnost, tedy možnost upravit parametry generování (jako například výška budov, hustota zastavění nebo různé vlastnosti terénu) a~tím dosáhnout požadovaných výsledků.

Samotné procedurální generování (myšleno vytváření zobrazitelných objektů) je zde možné rozdělit na několik částí\,--\,generování terénu, silnic a~budov.
K~vytvoření tohoto nástroje ovšem nebude využito pouze principů procedurálního generování\,--\,ale také principů počítačové grafiky, práce s knihovnou OpenGL i~lineární algebry.

\todo{Vlastní motivace - proč?}
Pro mě osobně byla počítačová grafika vždy velmi zajímavá a přitažlivá.


\pagebreak
\section{Obsah kapitol}
%\todo{Struktura práce - co se kde probírá (odkazy na kapitoly, strany)}
Seznámení se s~principy, které jsou zapotřebí k~vytvoření výše popisovaného nástroje, proběhne v~kapitole \ref{ch:teorie} na straně \pageref{ch:teorie}.
Jedná se například o~práci
s~OpenGL (kapitola \ref{ch:teorie:opengl}, strana \pageref{ch:teorie:opengl}),
základy lineární algebry (kapitola \ref{ch:teorie:lineární-algebra}, strana \pageref{ch:teorie:lineární-algebra}),
použité šumové funkce (kapitola \ref{ch:teorie:perlin}, strana \pageref{ch:teorie:perlin})
nebo principy práce s~texturami (kapitola \ref{ch:teorie:textury}, strana \pageref{ch:teorie:textury}).

\todo{O co jde v návrhu?}
V kapitole \ref{ch:navrh} na straně \pageref{ch:navrh} \dots
\todo{Konkrétní ukázka na podkapitoly.}

Kapitola \ref{ch:impl} na straně \pageref{ch:impl} bude pojednávat o~konkrétní implementaci všech výše zmíněných modulů programu.
\todo{Konkrétní ukázka na podkapitoly.}

V kapitole \ref{ch:závěr} na straně \pageref{ch:závěr} bude krátce zhodnocena vzniklá implementace nástroje, co s~tímto nástrojem lze a co nelze.
Dále budou popsány i~možnosti rozšíření


%===========================================================================================
%	KAPITOLA TEORIE
%
% - 40-50% obsahu práce
% - vhodné psát vlastními slovy, obsahově může kopírovat literaturu
% - trohu znalý člověk z oblasti by měl být schopen vše pochopit
%===========================================================================================
\chapter{\todo{Teorie}}\label{ch:teorie}
%\todo{Co je obsahem této sekce...}
Tato kapitola slouží k~seznámení se se základy všech různých odvětví, jejichž znalost implementace tohoto nástroje vyžaduje.


\section{OpenGL}\label{ch:teorie:opengl}
\todo{Co je to OpenGL, k čemu se používá?}

\subsection{Vertex}\label{ch:teorie:opengl:vertex}
\todo{Co je to vertex?}

\begin{figure}[H]
	\centering
	\begin{subfigure}{0.32\textwidth}
		\includegraphics[width=\textwidth]{obrazky-figures/placeholder.pdf}
		\caption{Pozice}
		\label{fig:opengl:vertex:p}
	\end{subfigure}
	\hfill
	\begin{subfigure}{0.32\textwidth}
		\includegraphics[width=\textwidth]{obrazky-figures/placeholder.pdf}
		\caption{Pozice, Normála}
		\label{fig:opengl:vertex:pn}
	\end{subfigure}
	\hfill
	\begin{subfigure}{0.32\textwidth}
		\includegraphics[width=\textwidth]{obrazky-figures/placeholder.pdf}
		\caption{Pozice, Normála, Barva}
		\label{fig:opengl:vertex:pnb}
	\end{subfigure}
	\caption{Ukázka vertexů s různými atributy}
	\label{fig:opengl:vertex}
\end{figure}

\todo{K čemu se používá?}
\todo{Jak a kde se používá?}

\subsubsection*{Vertex atribut}\label{ch:teorie:opengl:vertex:attr}
\todo{Co je to vertex?}
\todo{Jak a kde se používá?}

\subsection{Index}\label{ch:teorie:opengl:index}
\todo{Co je to index?}
\todo{Jak a kde se používá?}

\subsection{Shadery}\label{ch:teorie:opengl:shadery}
\todo{Co je to shader?}
\todo{Jak a kde se používá?}

\subsubsection*{Vertex shader}\label{ch:teorie:opengl:vertex-shader}
\todo{Co konkrétně dělá vertex shader?}

\begin{figure}[H]
	\centering
	\includegraphics[width=.5\linewidth]{obrazky-figures/placeholder.pdf}
	\caption{Demonstrace funkce vertex shaderu}
	\label{fig:opengl:vertex-shader}
\end{figure}

\subsubsection*{Fragment shader}\label{ch:teorie:opengl:fragment-shader}
\todo{Co konkrétně dělá fragment shader?}

\begin{figure}[H]
	\centering
	\includegraphics[width=.5\linewidth]{obrazky-figures/placeholder.pdf}
	\caption{Demonstrace funkce fragment shaderu}
	\label{fig:opengl:fragment-shader}
\end{figure}

\subsubsection*{Stínovací modely}\label{ch:teorie:opengl:shading}
\todo{Co jsou to stínovací modely, k čemu slouží a jak jsou použity?}

\todo{Phong}

\begin{figure}[H]
	\centering
	\begin{subfigure}{0.24\textwidth}
		\includegraphics[width=\textwidth]{obrazky-figures/placeholder.pdf}
		\caption{Ambient}
		\label{fig:opengl:phong:ambient}
	\end{subfigure}
	\hfill
	\begin{subfigure}{0.24\textwidth}
		\includegraphics[width=\textwidth]{obrazky-figures/placeholder.pdf}
		\caption{Diffuse}
		\label{fig:opengl:phong:diffuse}
	\end{subfigure}
	\hfill
	\begin{subfigure}{0.24\textwidth}
		\includegraphics[width=\textwidth]{obrazky-figures/placeholder.pdf}
		\caption{Specular}
		\label{fig:opengl:phong:specular}
	\end{subfigure}
	\hfill
	\begin{subfigure}{0.24\textwidth}
		\includegraphics[width=\textwidth]{obrazky-figures/placeholder.pdf}
		\caption{Result}
		\label{fig:opengl:phong:result}
	\end{subfigure}
	\caption{Ukázka jednotlivých částí Phongovy rovnice}
	\label{fig:opengl:phong}
\end{figure}

\todo{Ukázka dalších.}


\section{Lineární algebra}\label{ch:teorie:lineární-algebra}
\todo{XXXXXXXXXXXXXXXXXXXXXX}

\subsection{Vektory}\label{ch:teorie:lineární-algebra:vektory}
\todo{Všeobecně co jsou to vektory?}
\todo{K čemu jsou vektory v počítačové grafice dobré?}

\begin{figure}[H]
	\centering
	\includegraphics[width=.33\linewidth]{obrazky-figures/placeholder.pdf}
	\caption{Vektor}
	\label{fig:lineární-algebra:vektor}
\end{figure}

\subsubsection*{Sčítání}
\todo{Jak funguje sčítání vektorů?}
\todo{Kdy je užitečné a kdy se používá? Příklad.}

\begin{figure}[H]
	\centering
	\includegraphics[width=.33\linewidth]{obrazky-figures/placeholder.pdf}
	\caption{Sčítání vektorů}
	\label{fig:lineární-algebra:sčítání}
\end{figure}

\subsubsection*{Odčítání}
\todo{Jak funguje odečítání vektorů?}
\todo{Kdy je užitečné a kdy se používá? Příklad.}

\begin{figure}[H]
	\centering
	\includegraphics[width=.33\linewidth]{obrazky-figures/placeholder.pdf}
	\caption{Odčítání vektorů}
	\label{fig:lineární-algebra:odčítání}
\end{figure}

\subsubsection*{Skalární součin}
\todo{Co je to skalární součin a jak to funguje?}
\todo{Kdy je užitečné a kdy se používá? Příklad.}

\begin{figure}[H]
	\centering
	\includegraphics[width=.33\linewidth]{obrazky-figures/placeholder.pdf}
	\caption{Skalární součin}
	\label{fig:lineární-algebra:dot}
\end{figure}

\subsubsection*{Vektorový součin}
\todo{Co je to vetorový součin a jak to funguje?}
\todo{Kdy je užitečné a kdy se používá? Příklad.}

\begin{figure}[H]
	\centering
	\includegraphics[width=.33\linewidth]{obrazky-figures/placeholder.pdf}
	\caption{Vektorový součin}
	\label{fig:lineární-algebra:cross}
\end{figure}

\subsection{Průsečíky}\label{ch:teorie:lineární-algebra:průsečíky}
\todo{XXXXXXXXXXXXXXXXXXXXXX}


\section{Perlinův šum}\label{ch:teorie:perlin}
\todo{Co je to Perlinův šum a čím je speciální?}
\todo{Jaké existují další typy šumů?}

\begin{figure}[H]
	\centering
	\begin{subfigure}{0.32\textwidth}
		\includegraphics[width=\textwidth]{obrazky-figures/placeholder.pdf}
		\caption{Náhodný šum}
		\label{fig:perlin:porovnání:náhodný}
	\end{subfigure}
	\hfill
	\begin{subfigure}{0.32\textwidth}
		\includegraphics[width=\textwidth]{obrazky-figures/placeholder.pdf}
		\caption{Perlinův šum}
		\label{fig:perlin:porovnání:perlin}
	\end{subfigure}
	\hfill
	\begin{subfigure}{0.32\textwidth}
		\includegraphics[width=\textwidth]{obrazky-figures/placeholder.pdf}
		\caption{Simplexní šum}
		\label{fig:perlin:porovnání:simplex}
	\end{subfigure}
	\caption{Porovnání náhodného, Perlinova a simplexního šumu}
	\label{fig:perlin:porovnání}
\end{figure}


\section{Principy procedurálního generování}\label{ch:teorie:generování}
\todo{XXXXXXXXXXXXXXXXXXXXXX}


\section{Textury}\label{ch:teorie:textury}
\todo{XXXXXXXXXXXXXXXXXXXXXX}


%===========================================================================================
%	KAPITOLA NÁVRHU ŘEŠENÍ NÁSTROJE / SHRNUTÍ DOSAVADNÍHO STAVU OBLASTI
%
% - smyslem je na základě zhodnocení současného stavu určit cíl práce
% - obecně platný (není konkrétní k C++ či OpenGL, ...)
% - odkazy na teoretickou část
%===========================================================================================
\chapter{Návrh řešení}\label{ch:navrh}

\todo{Co už existuje, co už je vyřešeno.}

\todo{Co by bylo vhodné vyřešit.}

\todo{Co to má dělat?}
\todo{Jaké to má mít parametry?}


%===========================================================================================
%	KAPITOLA IMPLEMENTACE NÁSTROJE
%
% - co tvoří podstatu vlastní práce
% - jak a jakými prostředky bylo dílo vytvořeno
% - jaké jsou výsledky práce
% - veškeré technické detaily, které nejsou podstatné k pochopení podstaty práce patří do příloh
%===========================================================================================
\chapter{Implementace}\label{ch:impl}

\todo{Blokové schéma, objektový návrh, seznam tříd, struktura SW...}

\todo{Volba OS...}
\todo{Volba programovacího jazyka...}
\todo{Volba knihoven...}

\todo{Omezení, problémy SW...}


\section{Použité knihovny}\label{ch:impl:knihovny}

\begin{description}
	\item[geGL]
	\todo{ovládání OpenGL}
	
	\item[ArgumentViewer]
	\todo{zpracování příkazů}
	
	\item[BasicCamera]
	\todo{snadná správa kamery}
	\todo{transformační matice, \dots}
	
	\item[SDL2CPP]
	\todo{SDL wrapper pro C++}
	
	\item[Vars]
	\todo{správa proměnných napříč programem}
	
	\item[glm]
	\todo{matematická knihovna}
	\todo{"implementuje" funkce z shaderů}
	
	\item[SDL2]
	\todo{komunikace s Windows API}
	\todo{sprava okna}
	\todo{správa input (keyboard, mouse)}
	
	\item[FreeImage, FreeImagePlus]
	\todo{načítání obrázků}
\end{description}


\section{Zobrazení}\label{ch:impl:zobrazení}
\todo{Jak probíhá vykreslování? \ref{ch:teorie:opengl:shadery}}
\todo{Klíčové vlastnosti?}

\subsection{Phongovo stínování}\label{ch:impl:zobrazení:phong}
\todo{Co je to Phongovo stínování odkaz do teorie \ref{ch:teorie:opengl:shading}.}

\todo{Jak je implementováno v HLSL.}


\section{Terén}\label{ch:impl:terén}
\todo{Jaký je cíl, co má být výsledkem generování?}

\todo{Potřebné třídy pro generování - map, chunk, heightMap}

\todo{Samotný proces generování}
\todo{vytváření vrcholů}
\todo{vytváření indexů}

\begin{figure}[H]
	\centering
	\begin{subfigure}{0.32\textwidth}
		\includegraphics[width=\textwidth]{obrazky-figures/placeholder.pdf}
		\caption{Vrcholy v prostoru}
		\label{fig:terén:vrcholy}
	\end{subfigure}%
	\hspace{0.02\textwidth}%
	\begin{subfigure}{0.32\textwidth}
		\includegraphics[width=\textwidth]{obrazky-figures/placeholder.pdf}
		\caption{Propojení vrcholů indexy}
		\label{fig:terén:trojůhelníky}
	\end{subfigure}
	\caption{Ukázka postupu vytvoření modelu terénu}
	\label{fig:terén}
\end{figure}

\todo{vytváření normál}

\begin{figure}[H]
	\centering
	\begin{subfigure}{0.32\textwidth}
		\includegraphics[width=\textwidth]{obrazky-figures/placeholder.pdf}
		\caption{Výběr blízkých vrcholů}
		\label{fig:terén:normály:vektory}
	\end{subfigure}
	\hfill
	\begin{subfigure}{0.32\textwidth}
		\includegraphics[width=\textwidth]{obrazky-figures/placeholder.pdf}
		\caption{Průběžné výpočty vektorů}
		\label{fig:terén:normály:výpočet}
	\end{subfigure}
	\hfill
	\begin{subfigure}{0.32\textwidth}
		\includegraphics[width=\textwidth]{obrazky-figures/placeholder.pdf}
		\caption{Výsledná normála}
		\label{fig:terén:normály:výsledek}
	\end{subfigure}
	\caption{Ukázka výpočtu normály částí terénu}
	\label{fig:terén:normály}
\end{figure}

\todo{Použité struktury}

\todo{Existující problémy, podněty pro zlepšení?}


\section{Silnice}\label{ch:impl:silnice}
\todo{Jaký je cíl, co má být výsledkem generování?}

\todo{Potřebné třídy pro generování - streetMap, street, heightMap, map}

\todo{Samotný proces generování \ref{fig:silnice}}
\todo{postup silnic \ref{fig:silnice:postup}}
\todo{vznik kolizí \ref{fig:silnice:kolize}}
\todo{vytváření podsilnic \ref{fig:silnice:nové}}
\todo{rozšiřování silnic ("prodlužování")}

\begin{figure}[H]
	\centering
	\begin{subfigure}{0.32\textwidth}
		\includegraphics[width=\textwidth]{obrazky-figures/placeholder.pdf}
		\caption{Postup silnic prostorem}
		\label{fig:silnice:postup}
	\end{subfigure}
	\hfill
	\begin{subfigure}{0.32\textwidth}
		\includegraphics[width=\textwidth]{obrazky-figures/placeholder.pdf}
		\caption{Kolize silnic}
		\label{fig:silnice:kolize}
	\end{subfigure}
	\hfill
	\begin{subfigure}{0.32\textwidth}
		\includegraphics[width=\textwidth]{obrazky-figures/placeholder.pdf}
		\caption{Vznik nových silnic}
		\label{fig:silnice:nové}
	\end{subfigure}
	\caption{Ukázka postupu pro generování silnic}
	\label{fig:silnice}
\end{figure}

\todo{výpočet kolizí (zmínka o QuadTree \ref{ch:impl:silnice:quadtree})}
\todo{ukládání kolizí a práce s nimi}

\todo{Existující problémy, podněty pro zlepšení?}

\subsection{QuadTree}\label{ch:impl:silnice:quadtree}
\todo{Co je to QuadTree a k čemu je použit?}

\todo{Implementace pro hledání bodů v prostoru.}
\todo{Popis obrázku \ref{fig:silnice:quadtree}.}

\begin{figure}[H]
	\centering
	\begin{subfigure}{0.32\textwidth}
		\includegraphics[width=\textwidth]{obrazky-figures/placeholder.pdf}
		\caption{Jeden bod v prostoru}
		\label{fig:silnice:quadtree:1}
	\end{subfigure}
	\hfill
	\begin{subfigure}{0.32\textwidth}
		\includegraphics[width=\textwidth]{obrazky-figures/placeholder.pdf}
		\caption{Dva body v prostoru}
		\label{fig:silnice:quadtree:2}
	\end{subfigure}
	\hfill
	\begin{subfigure}{0.32\textwidth}
		\includegraphics[width=\textwidth]{obrazky-figures/placeholder.pdf}
		\caption{Čtyři body v prostoru}
		\label{fig:silnice:quadtree:4}
	\end{subfigure}
	\caption{Ukázka dělení prostoru pomocí QuadTree pro body}
	\label{fig:silnice:quadtree}
\end{figure}

\todo{Princip fungování upravené verze QuadTree.}
\todo{Obal objektu RectBounds.}

\begin{figure}[H]
	\centering
	\begin{subfigure}{0.32\textwidth}
		\includegraphics[width=\textwidth]{obrazky-figures/placeholder.pdf}
		\caption{Jeden objekt v prostoru}
		\label{fig:silnice:quadtree-obj:1}
	\end{subfigure}
	\hfill
	\begin{subfigure}{0.32\textwidth}
		\includegraphics[width=\textwidth]{obrazky-figures/placeholder.pdf}
		\caption{Dva body v prostoru}
		\label{fig:silnice:quadtree-obj:2}
	\end{subfigure}
	\hfill
	\begin{subfigure}{0.32\textwidth}
		\includegraphics[width=\textwidth]{obrazky-figures/placeholder.pdf}
		\caption{Čtyři body v prostoru}
		\label{fig:silnice:quadtree-obj:4}
	\end{subfigure}
	\caption{Ukázka dělení prostoru pomocí QuadTree pro objekty}
	\label{fig:silnice:quadtree-obj}
\end{figure}

\todo{Přesná implementace QuadTree pro objekty v obálce RectBounds.}


\section{Parcely}\label{ch:impl:parcely}
\todo{Jaký je cíl, co má být výsledkem generování?}

\subsection{Parcely pro budovy}\label{ch:impl:parcely:budovy}
\todo{parcely pro budovy}

\todo{Potřebné třídy pro generování - streetMap, street}

\todo{Samotný proces generování}
\todo{průchod přes silnice}
\todo{vyhledání a použití "křižovatek"}
\todo{neduplicita parcel}

\begin{figure}[H]
	\centering
	\begin{subfigure}{0.32\textwidth}
		\includegraphics[width=\textwidth]{obrazky-figures/placeholder.pdf}
		\caption{Vytvoření parcely}
		\label{fig:parcely:budovy:vytváření:první}
	\end{subfigure}
	\hfill
	\begin{subfigure}{0.32\textwidth}
		\includegraphics[width=\textwidth]{obrazky-figures/placeholder.pdf}
		\caption{Uložení částí cest}
		\label{fig:parcely:budovy:vytváření:uložení}
	\end{subfigure}
	\hfill
	\begin{subfigure}{0.32\textwidth}
		\includegraphics[width=\textwidth]{obrazky-figures/placeholder.pdf}
		\caption{Vytvoření dalších parcel}
		\label{fig:parcely:budovy:vytváření:další}
	\end{subfigure}
	\caption{Ilustrace vytváření parcel pro budovy}
	\label{fig:parcely:budovy:vytváření}
\end{figure}

\subsubsection*{Dělení parcel}\label{ch:impl:parcely:budovy:dělení}

\begin{figure}[H]
	\centering
	\begin{subfigure}{0.32\textwidth}
		\includegraphics[width=\textwidth]{obrazky-figures/placeholder.pdf}
		\caption{Samotná parcela v prostoru}
		\label{fig:parcely:budovy:dělení:samotná}
	\end{subfigure}
	\hfill
	\begin{subfigure}{0.32\textwidth}
		\includegraphics[width=\textwidth]{obrazky-figures/placeholder.pdf}
		\caption{Rozdělení na „pod-parcely”}
		\label{fig:parcely:budovy:dělení:subparcely}
	\end{subfigure}
	\hfill
	\begin{subfigure}{0.32\textwidth}
		\includegraphics[width=\textwidth]{obrazky-figures/placeholder.pdf}
		\caption{Náhodné seskupení}
		\label{fig:parcely:budovy:dělení:seskupení}
	\end{subfigure}
	\caption{Ilustrace zpracování parcel\,--\,vytváření míst pro jednotlivé budovy}
	\label{fig:parcely:budovy:dělení}
\end{figure}

\subsection{Parcely pro silnice}\label{ch:impl:parcely:silnice}
\todo{parcely pro silnice}

\todo{Potřebné třídy pro generování - streetMap, street}

\todo{Samotný proces generování}
\todo{průchod přes silnice}
\todo{vyhledání a použití "křižovatek"}

\begin{figure}[H]
	\centering
	\begin{subfigure}{0.32\textwidth}
		\includegraphics[width=\textwidth]{obrazky-figures/placeholder.pdf}
		\caption{Vytvoření parcely}
		\label{fig:parcely:silnice:vytváření:první}
	\end{subfigure}%
	\hspace{0.02\textwidth}%
	\begin{subfigure}{0.32\textwidth}
		\includegraphics[width=\textwidth]{obrazky-figures/placeholder.pdf}
		\caption{Vytvoření dalších parcel}
		\label{fig:parcely:silnice:vytváření:další}
	\end{subfigure}
	\caption{Ilustrace vytváření parcel pro cesty}
	\label{fig:parcely:silnice:vytváření}
\end{figure}


\section{Budovy}\label{ch:impl:budovy}
\todo{Jaký je cíl, co má být výsledkem generování?}
\todo{typy budov}

\todo{Potřebné třídy pro generování - parcel, building, buildingPart, heightMap}

\todo{Samotný proces generování}
\todo{vytváření kvádrů/jiných tvarů}
\todo{jejich posun/orientace}
\todo{generování normál}


\section{Textury}\label{ch:impl:textury}
\todo{Textury?}

\subsection{Térén}\label{ch:impl:textury:terén}
\todo{Vybrané textury pro terén}

\todo{Způsob mapování textur na terén - shader.}

\begin{figure}[H]
	\centering
	\begin{subfigure}{0.49\textwidth}
		\includegraphics[width=\textwidth]{obrazky-figures/placeholder.pdf}
		\caption{Zvýrazněné části terénu určené texturám}
		\label{fig:textury:terén:místa}
	\end{subfigure}
	\hfill
	\begin{subfigure}{0.49\textwidth}
		\includegraphics[width=\textwidth]{obrazky-figures/placeholder.pdf}
		\caption{Výsledné mapování textur}
		\label{fig:textury:terén:výsledek}
	\end{subfigure}
	\caption{Vizualizace mapování textur na model terénu}
	\label{fig:textury:terén}
\end{figure}

\subsection{Budovy}\label{ch:impl:textury:budovy}
\todo{Generování textury "podkladu".}\todo{Význam a využití.}

\todo{Generování textury oken.}\todo{Význam a využití.}\todo{Reflekce - shader.}

\todo{Způsob mapování textury na terén - shader.}

\subsection{Prostředí}
\todo{Skybox?}


%===========================================================================================
%	ZÁVĚREČNÁ KAPITOLA
%
%===========================================================================================
\chapter{Závěr}\label{ch:závěr}

\todo{Krátké shrnutí závěru práce...}

\todo{Jak byl záměr práce splněn?}
\todo{Zhodnocení dosažených výsledků.}

\todo{Co nebylo implementováno, možnost pokračování, výhled do budoucna.}

\todo{Osobní dojmy - co mi práce dala?}
